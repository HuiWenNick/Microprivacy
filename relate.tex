\section{Related Work}


Privacy is recognized as a dominant concerning factor for a mobile device, in particular for smartphones.
Several pieces of work have studied how to protect user privacy by detecting  behaviors   accessing sensitive data.
TapPrints~\cite{tapprints2012} uses motion sensors (i.e, accelerometer and gyroscope) to infer the location of taps on the touch screens and detect smartphone-sensitive data such as password.
TaintDroid~\cite{taintdroid2014} provides an Android's virtualized execution environment to analyze and determine whether third-party applications leak out user  privacy data or not.
Similarly,  PiOS~\cite{pios2011} conducts a static analysis over iOS application intermediate code to discover any possible leaks of sensitive information by third parties. EnCore~\cite{encore2014} introduces a privacy preserving approach for mobile social applications, where people encounter strangers within a wide range of communication.
In contrast, our work focuses on a totally different privacy on smartphones, micro-privacy on screen.
A light-emitting phone screen maybe leak  out phone user's privacy information in the public area.

To protect user privacy, it is important to distinguish from a malicious behavior.
RiskRanker~\cite{riskranker2012} identifies and assesses potential security risks from  untrusted apps to defend against malware.
MAdFraud~\cite{madfraud2014} studies advertisement fraud issues that is displayed on app's GUI and is embedded with Ad libraries from  the ad provider on mobile phones.
Differently, our work proposes to preserve user privacy via issuing an alert/alarm to users, where the front camera acts as an eye on the back.
The front camera identifies the scene that a phone user's private information might be leaking out.


Similar to our work, several smartphone applications also adopt the front camera for different purposes. 
Carsafe~\cite{carsafe2013} protects a  driver safety by alerting drivers to dangerous driving conditions and behavior. It uses  the front camera to monitor the driver  status and the rear camera to detect road conditions.
Our work also uses the front camera, but we use it to detect visual eavesdroppers to protect user privacy.
ViRi~\cite{viri2013} restores the front-view effect at the  slanted viewing angle to help users  see smartphone screen content conveniently. It uses the front camera  to  determine whether the screen is needed to slant a certain angle and adds  a commercial off-the-shelf fisheye lens to address  the limited field view of the front camera.
Our work comes from a different perspective. it is common that people  carry and use their phones when they are in public areas. It is impractical and inconvenient to require a user to pre-setup a fisheye lens  on the front camera to detect visual eavesdroppers. Therefore, our approach does not depend on the addition of an extra hardware. 